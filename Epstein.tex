\documentclass[11pt]{article} % type de document avec taille de police par défaut
% =============================================================
\usepackage{geometry} % la "géométrie" de la page
\geometry{a4paper, top=2.1cm, bottom=2.1cm, left=2.1cm, right=2.1cm}
% --------------- XeLaTeX spécifique
\usepackage{fontspec} % pour le bon encodage des PDFs => police lmodern
\usepackage[francais]{babel} % pour franciser LaTeX
  \frenchbsetup{StandardLayout=true}
% ---------------
\setcounter{secnumdepth}{0}  % sans numérotation des sections
\setlength{\parindent}{0pt} % no indent
\usepackage{parskip}
  \setlength{\parskip}{.5\baselineskip}
\usepackage{titlesec}
\titlespacing*{\section}
  {0pt}{2.1ex plus 1ex minus .2ex}{1.4ex plus .4ex minus .3ex}
\titlespacing*{\paragraph}
  {0pt}{0ex}{.7em plus .4em minus .3em}
% ---------------
\usepackage[shortlabels]{enumitem}
% ---------------
\usepackage[pdfencoding=auto,bookmarks=false]{hyperref} % pour les liens. En règle général il doit être le dernier package chargé.
\hypersetup{
  pdftitle={Parents d'adolescents, Professeurs d'adolescents, L'adolescence en question},
  pdfauthor={Jean Epstein}
}
%==============================================================

% ---------------
\title{« Parents d'adolescents, Professeurs d'adolescents\\L'adolescence en question »}
\author{Jean Epstein}
\date{Conférence\footnote{La conférence a été organisée par la Commission École/Familles du Collège Louise Michel à Chaumont, en partenariat avec la Caisse d’Allocations Familiales.}\ \ du 14 novembre 2003}
% ---------------

\begin{document}
%==============================================================
\maketitle


% =============================================================
\section{MARCHER TOUS ENSEMBLE}
% =============================================================

  Dans le milieu scolaire, la culture partenariale autour du jeune est extrêmement importante à développer, mais elle ne fait pas encore partie des habitudes.\newline
  D'une part, l'école, les enseignants et tout le personnel présent dans l'établissement scolaire, sont les mieux placés pour observer un certain nombre de difficultés chez des jeunes. Mais en même temps, ils doivent savoir qu'ils n'ont pas la toute puissance de l'éducation, ils ne sont pas responsables de tout: ils peuvent constater, et tenter d'en parler aux parents, sans que leur intervention soit suivie d'un résultat au sein de la famille. Toutefois, il serait regrettable de faire abstraction du fait que l'enfant arrive en classe avec des problématiques familiales.\newline
  Par ailleurs, les parents sont parfois amenés à se confier aux enseignants, mais l'école ne peut pas se mêler de tout. Les parents restent les premiers éducateurs de leurs enfants, et l'école est un lieu pédagogique.\newline
  Voici un exemple. Un parent va parler de ses difficultés conjugales à un enseignant. Que va faire l'enseignant ? S'il dit: \textit{« Moi, c'est seulement l'élève qui me concerne »}, c'est dommage parce que l'élève est sûrement très concerné par cette problématique. Inversement, si l'enseignant dit: \textit{« Vous tombez à pic. Moi, je suis professeur, mais parallèlement, je suis conseiller conjugal, psychologue, sexologue, je fais agence pour l'emploi le mardi et, voyez, ce canapé est un divan, allongez-vous et racontez-moi votre enfance »}\dots\ La réponse est plutôt risquée.

  C'est pourquoi une relation de confiance doit se tisser entre les différents acteurs qui entourent l'enfant: parents et enseignants sont parfaitement complémentaires.\newline
  Il est évident que nous devons marcher tous ensemble autour et avec les jeunes dans le sens d'une relation inter-professionnels, inter-acteurs, en n'ayant pas peur de perdre des pouvoirs, en mettant en commun des territoires.


% =============================================================
\section{VOUS ÊTES LES MEILLEURS PARENTS DU MONDE\dots}
% =============================================================

  Vous allez entendre parler ce soir de la construction des repères chez l'adolescent. Vous êtes parent ou enseignant (ou les deux), et face à tel ou tel point qui sera abordé, vous allez culpabiliser en pensant \textit{« Oh là là\dots\ Cela, je l'ai fait\dots\ Cela, je ne l'ai pas fait\dots »}. Au lieu de raisonner ainsi, demandez-vous plutôt: \textit{« Moi, en tant que parent »} ou \textit{« Moi, en tant qu'enseignant\dots\ Qu'est-ce que je fais, et qu'est-ce que je peux faire pour aider ce jeune à construire tel ou tel repère ? »}.\newline
  Je suis moi-même parent et psychologue. En tant que psychologue, j'ai de très bonnes théories, mais avec mes enfants, elles n'ont pas du tout marché\dots\ Donc, le psychologue qui va vous parler ce soir, va vous présenter les conclusions de ses recherches. Et puis après, vous pourrez faire ce que vous voudrez. Je demeure intimement convaincu que vous êtes les meilleurs parents du monde\dots\ vu que vos enfants n'en ont pas d'autres. Cela ne veut pas dire que vous êtes bons parents, cela signifie que c'est avec ces parents-là qu'ils vont devoir se construire. Et il va falloir les y aider. C'est alors que le mot \textit{confiance }prend tout son sens: en tant que parent, il est indispensable d'avoir confiance en soi pour aider nos jeunes à avoir, bien évidemment, confiance en eux.


% =============================================================
\section{L'EDUCATIF ET LE PEDAGOGIQUE SONT COMPLEMENTAIRES}
% =============================================================

  D'un côté, l'enseignant, le professionnel, doit se demander ce qu'il peut faire pour aider le jeune à trouver ses repères.\newline
  De l'autre, les parents doivent avoir confiance en eux afin que leur enfant puisse lui aussi développer cette confiance en soi.\newline
  Pendant longtemps, on a séparé l'éducatif du pédagogique: un travail de recherche a permis de démontrer que des apprentissages d'ordre pédagogique ne peuvent s'effectuer sans certains apprentissages éducatifs.\newline
  J'ai participé à une étude menée au Québec sur le syndrome hyperactif, né aux États-Unis, et apparu récemment en France. On pensait que c'était une maladie neurologique, traitée avec une molécule chimique (la Ritaline), à tel point que des parents réclamaient des distributeurs de ce médicament dans les rues. Des recherches plus pertinentes ont alors montré que seulement un petit nombre des enfants observés avait des problèmes d'ordre pathologique.\newline
  En réalité, un lien a été établi entre l'hyperactivité des enfants et le mode alimentaire familial: chaque membre de la famille se restaurait en « électron libre » à des moments différents, sans jamais se retrouver ensemble autour d'un repas. Ainsi, le jeune en construction devenait incapable de rester assis et les parents incapables de lui imposer cette frustration pourtant bénéfique.\newline
  Car en effet, comment un enfant pourra-t-il apprendre à lire s'il n'est pas capable de rester assis ? On retrouve chez l'enfant hyperactif des difficultés dans l'apprentissage de la lecture, il peut aussi devenir un enfant « zapping » qui passe sans cesse d'une activité à une autre.\newline
  On observe ici que des adolescents arrivent à un âge très avancé sans avoir acquis un certain nombre d'apprentissages de la petite enfance. L'éducatif et le pédagogique sont donc bien complémentaires.


% =============================================================
\section{BEBES, ADOLESCENTS, MÊME COMBAT !}
% =============================================================

  Des enfants arrivent donc à l'âge de l'adolescence sans avoir acquis certains apprentissages de la petite enfance. Mais, pour autant, il ne faut pas penser que « tout se joue avant 6 ans, ou avant 3 ans ».Il convient plutôt de dire que tout se joue\dots\ avant la mort !\newline
  Avec l'adolescent, des « sessions de rattrapage » sont heureusement toujours possibles ! On peut l'aider à progresser pour re-construire ces apprentissages.\newline
  Il y a peu de différences entre le bébé et l'adolescent. Tous deux se posent la même question fondamentale: \textit{« Est-ce que mes parents m'aiment tel que je suis ? »} L'adolescent ajoute une deuxième question tout aussi importante: \textit{« Est-ce bien vrai que je suis tombé sur les plus nuls ?\dots »} Et dans les deux cas, la réponse doit être positive: \textit{« Je suis tombé sur les plus nuls, mais est-ce qu'ils m'aiment ? »} L'adolescent, c'est simple\dots: \textit{« J'ai besoin que mes parents me donnent des conseils\dots\ pour ne pas les suivre. Si mes parents attendent que je les suivent, ils sont mal partis ! Par contre, s'ils ne m'en donnent pas, c'est moi qui suis mal parti ! »}\newline
  Que l'on soit parent ou professionnel, il faut toujours essayer d'entendre ce que l'adolescent nous dit. Selon les jours, cet enfant parle en tant qu'adulte en devenir (bien plus adulte qu'on ne pouvait imaginer qu'il le soit, ou, bien plus adulte qu'on ne pouvait l'être à son âge). D'autres jours, c'est le bébé qui se trouve en face de nous (bien plus bébé qu'on ne peut imaginer qu'il le soit).\newline
  L'exemple des fréquents maux de ventre chez l'adolescent serait intéressant à étudier et c'est aussi pourquoi le partenariat entre la santé scolaire et les enseignants est à développer. Ainsi, les jours de maux de ventre, soit nous entendons l'adulte, et nous répondons en adulte: \textit{« Qu'est-ce que tu as encore mangé ? »}\dots\ Soit nous entendons le bébé qui se cache à l'intérieur, et nous traduisons sa demande qui est: \textit{« J'ai envie que tu t'occupes de moi »}. C'est donc, selon les jours, le petit ou le grand qui s'exprime\dots à nous d'essayer de ne pas se tromper de jour (1\textsuperscript{er} conseil)\dots\ mais aussi d'essayer d'entendre ce qu'il nous demande (2\textsuperscript{ème} conseil).


% =============================================================
\section{PAPA-OURS ET MAMAN-OURS}
% =============================================================

  L'enfance doit être une période de construction de repères.\newline
  Depuis quelques années, tout s'est précipité pour la famille. Prenons comme exemple le travail de la femme: en 1970, 24\,\% des mères de 2 enfants travaillaient hors-foyer; en 1990: 77\,\% !\newline
  Le père et la mère travaillent donc, le schéma « papa-ours = grand bol / maman-ours = moyen bol » n'a plus cours\dots\ Mais si l'être humain s'adapte à ces changements, les neurones, eux, vont plus lentement. Il leur faut parfois quatre générations pour intégrer les faits de société. Ainsi, les mamans qui travaillent culpabilisent encore à cause de leur manque de disponibilité, et n'osent plus assez dire non aux enfants.\newline
  En matière d'éducation, nous sommes passés du mode autorité/sanction (« papa-ours »), au mode autorité/séduction: \textit{« Est-ce que mes enfants m'aiment ? Je ne veux pas jouer le rôle de la méchante qui interdit tout »}, C'est un progrès au sens de l'autorité, mais quelle fragilité ! De ce fait, il y a des parallèles à faire entre l'évolution fabuleuse du travail des mamans et l'apparition de l'« enfant roi ». Celui-ci va développer toute une stratégie basée sur la fragilité des parents pour devenir « chef de famille », qu'il soit bébé ou adolescent.


% =============================================================
\section{ADOLESCENCE: DERNIERE STATION AVANT L'AUTOROUTE}
% =============================================================

  On parle aujourd'hui de pré-adolescence, de post-adolescence, voire d'« adulescence »\dots\ mais qu'est-ce que l'adolescence ?\newline
  En réalité, l'adolescence est plus une période d'hypersensibilité où beaucoup de repères s'articulent, qu'une période de crise. Des parents s'inquiètent parfois que leur enfant n'ait pas fait sa « crise d'adolescence », alors que celle-ci n'est pas obligatoire\dots\newline
  En fait, l'adolescent affine souvent les repères intégrés dans l'enfance, ou révèle ceux qui n'ont pas été acquis. C'est plutôt une période de fragilité où il va falloir accepter certaines contradictions.\newline
  Parmi celles-ci apparaît la difficulté à faire cohabiter deux principes.

  \paragraph{Le principe de plaisir:} \textit{« Je suis le centre du monde, j'ai tout ce que je veux tout de suite »}. Il s'agit du tout petit, et cela ne doit pas durer: l'adulte doit veiller à valoriser ses compétences, mais en même temps, il doit lui faire comprendre qu'il n'y a pas que lui au monde.

  \paragraph{Le principe de réalité:} \textit{« Je ne suis pas le centre du monde et je dois attendre »}. C'est le contraire du principe de plaisir: il apprend à se projeter, à s'engager dans l'action pour un plaisir qui n'arrivera qu'ultérieurement.\newline
  Aujourd'hui, beaucoup d'adolescents sont incapables de se projeter: c'est la pathologie de l'immédiateté due à la quasi-exclusivité du principe de plaisir en tant que mode éducatif. Il favorise l'apparition des phénomènes addictifs comme la toxicomanie (recherche du plaisir immédiat, toujours insatisfait puisqu'il ne dure pas), le surendettement (\textit{« Achetez tout de suite, vous paierez dans un an »}), et le décrochage scolaire (\textit{« A quoi ça sert d'apprendre ? »}\dots\ sentiment d'inutilité accentué par le contexte socio-économique actuel: \textit{« Même les personnes bardées de diplômes ne trouvent pas de travail, alors\dots »}), Pour ces derniers, leur seule ressource est d'adopter un comportement qui perturbe la classe, ce qui risque de les amener à être exclus. S'ils sont orientés par exemple vers un enseignement professionnel en rapport avec les choix qui leur correspondent, ils retrouvent du sens, le désir d'apprendre, et se remettent à progresser. C'est pourquoi cette notion de plaisir et de réalité est complètement essentielle.\newline
  L'adolescence est bien une période où va se construire \textit{dernière station avant l'autoroute }une cohabitation équilibrée entre le principe de plaisir et le principe de réalité.


% =============================================================
\section{CONFIANCE - COMPETENCE - COHERENCE}
% =============================================================

  Trois mots-clés sont à retenir. De la part des adultes, l'adolescent a besoin de \emph{confiance}, la reconnaissance de ses \emph{compétences}, et de \emph{cohérence}.

  \paragraph{La confiance:} \textit{« J'ai besoin que mes parents aient confiance en moi, et j'ai besoin d'avoir confiance en moi, plus que jamais. Il est évident que je vais être particulièrement réactif si on me témoigne du mépris »}. Mais comme noté plus haut, les parents doivent avoir confiance en eux afin que leur enfant puisse lui aussi développer cette confiance en soi.

  \paragraph{La compétence:}  L'enfant doit être valorisé dans ses compétences. Le système scolaire est trop basé sur la compétition et des problèmes se posent quand l'enfant ne « rentre pas dans le moule ». Je souhaite beaucoup de courage aux parents de ces enfants ! On a été formatés pour éclairer ce qui ne va pas -- au niveau des jeunes, mais aussi au niveau des parents. \textit{« Les parents démissionnent »} dit-on, mais je continue à penser qu'ils sont plutôt « démissionnés ». Il faut donc les « re-missionner » !\newline
  Les jeunes sont remplis de compétences, mais ce ne sont pas forcément celles requises par les programmes scolaires. A l'évidence et mathématiquement, plus on rétrécit les moules, moins il y en a qui rentrent dedans.\newline
  On a trop tendance à éclairer le handicap, les manques, plutôt que l'enfant lui-même. Boris Cyrulnik parle du concept de résilience qui n'est qu'une illustration de cette notion de compétence (lire ses recherches sur le sujet: \textit{Un merveilleux malheur, } puis \textit{Les vilains petits canards}, et enfin \textit{Le murmure des fantômes)}. C'est une psychiatre américaine qui a créé la notion de résilience où, pour avoir la capacité de s'en sortir, il faut garder la tête hors de l'eau notamment grâce à l'humour (voir le film \textit{« La vie est belle »} du cinéaste Roberto Benigni).

  \paragraph{La cohérence:}  Les jeunes ont besoin de cohérence autour d'eux. L'adulte doit être cohérent dans ses valeurs, ses affirmations et ses actes.\newline
  Cela signifie qu'au sein de la famille, le discours doit être à peu près cohérent. Si papa et maman, en tant qu'homme et femme, ne s'entendent plus vraiment et veulent utiliser l'enfant pour mettre l'autre en difficulté, on a une autoroute pour que l'adolescent reste adolescent.\newline
  Autre exemple en famille: pourquoi tant de jeunes ne parviennent-ils pas à quitter la maison ? (voir le film \textit{« Tanguy »} d'Etienne Chatiliez). Leurs parents ont tous à peu près le même profil: la cinquantaine, ayant beaucoup investi dans leur carrière professionnelle quand l'enfant était petit, ils voudraient rattraper le temps perdu 25 ans après\dots\ et le jeune devient le chef de famille.\newline
  A l'école: il faut là aussi de la cohérence en établissant par exemple un règlement précis auquel l'adolescent a participé, qu'il ait intégré et qui soit appliqué. De même, tous les adultes de la communauté scolaire éducative doivent respecter et faire respecter ce règlement de façon cohérente. Si un jour, un adulte réprimande un élève pour un acte commis, un autre adulte doit le lendemain en faire de même si l'enfant recommence. Comment le jeune pourrait-il accepter le lundi une loi contraire à celle du mardi ? Pendant cette période de l'adolescence qui est si fragile, il y a un besoin de cohérence maximum.

  Dans l'intérêt de l'enfant, confiance et cohérence entre la famille et l'école sont fondamentales. A l'inverse, si des relations de défiance et d'incohérence s'installent, le jeune n'aura plus qu'à les monter l'une contre l'autre.


% =============================================================
\section{LA CONSTRUCTIONDES REPERES}
% =============================================================

  Parents et professionnels doivent connaître les trois types de repères nécessaires au jeune pour se construire. Ces trois champs sont en permanence interactifs: champ individuel de compétences, repères sociaux, mais aussi repères familiaux.

  \paragraph{Les repères individuels:}  L'enfant naît avec des rythmes, des rites et des modes d'intelligence qui lui sont propres. Il y a des intelligences plutôt « logiques », ou plutôt « littéraires », mais il n'existe pas de standards d'intelligence. Et dès la petite enfance, ces intelligences vont s'exprimer.\newline
  Le système scolaire se veut trop souvent logique sans prendre suffisamment en compte les différences entre les enfants. Un enfant, qu'il soit petit ou adolescent, a besoin d'être évalué, mais en fonction de ses propres compétences: il ne devrait y avoir qu'une case à remplir sur les carnets d'évaluation, celle intitulée « en cours d'acquisition ». Si la famille et l'école n'ont pas su valoriser les compétences propres d'un enfant qui n'entre pas dans le moule, il est en danger en tant que futur adolescent. Pourquoi ? S'il est plutôt « littéraire »: petit, on le dira \textit{« imaginatif »; }à la maternelle, il va s'appeler \textit{« créatif »; }en grande section, il sera dit \textit{« rêveur »}. Et très vite, on le désignera comme \textit{« inattentif », }voire \textit{« turbulent »}. Résultat: cet enfant sera en grande difficulté pour se construire.\newline
  Si des enfants ou des adolescents n'ont pas particulièrement les compétences requises dans le cadre scolaire, les parents doivent bien sûr les inciter à aller plus loin dans leur scolarité, mais ils doivent aussi valoriser d'autres champs de compétences. Il n'y a rien de pire pour un jeune d'être pris dans le feu croisé négatif éducatif de l'école et de sa propre famille, en bonne entente ! Il faudrait en réalité une valorisation de tous les modes d'intelligence et de compétences.

  \paragraph{Les repères sociaux:}  Le jeune doit apprendre à accepter les règles et les lois, dans le principe de réalité (\textit{« Il n'y a pas que moi »}).

  \paragraph{Les repères familiaux:}  Dès la petite enfance, l'enfant doit trouver des réponses aux questions suivantes:\newline
  \textit{« Mes parents m'aiment-ils ? »} \newline
  \textit{« Qui suis-je ? Quelle est mon histoire ? »} \newline
  \textit{« Quelle est ma place dans la famille ? »}

  De façon basique, le rôle d'un enfant, quelque soit son âge, c'est d'être l'enfant de ses parents. A ce sujet, il ne faut jamais oublier qu'aucun enfant n'a vocation à être le conjoint de ses parents, ni le parent de ses parents, ni le thérapeute de ses parents.\newline
  Exemple d'Eric: enfant violent, dont les parents sont au chômage. Il part seul le matin alors que ses parents restent en pyjama à la maison, jouant avec sa Game Boy\dots:\textit{« Travaille bien en classe pour réussir plus tard, et n'oublie pas de ramener le pain »}. Le rôle d'Eric est d'être le parent de ses parents, d'où l'hyper agressivité en classe, notamment avec les plus petits, ceux qui ont le droit d'être des enfants.\newline
  Où faut-il agir ? Pas sur lui-même, mais dans son champ familial, pour redonner confiance à ses parents.\newline
  Autre exemple: quand le milieu scolaire véhicule une image négative sur telle ou telle famille, l'enfant de cette même famille sera repéré et redouté à priori. Il arrive à l'école avec plein de compétences, mais il sent qu'on ne l'y attend pas, il sent que les enseignants ne souhaitent pas l'avoir dans leur classe. Conséquence: il risque de reproduire à peu près ce que sa famille a produit dans l'école et cela n'étonnera personne. \textit{« Est-ce que ma famille a une image positive ? »}, peut-être que cette question d'identification est primordiale.


% =============================================================
\section{PORTRAITS D'ADOLESCENTS}
% =============================================================

  Le développement de la violence chez les jeunes est souvent provoqué par le manque de ces trois repères: individuels, sociaux et familiaux.\newline
  J'ai participé à des recherches en 1990 à la demande du Conseil Européen sur l'identification des facteurs de violence chez les jeunes. Elles ont mis en évidence 3 constats:
  \begin{itemize}[--]

    \item Les phénomènes d'auto-violence (toxicomanie, comportements à risque ou suicidaires, alcoolisme, et décrochage scolaire) sont plus fréquents que les comportements violents tournés vers l'extérieur.

    \item Il y a autant d'expression de violence chez les filles que chez les garçons.

    \item Milieu rural et milieu urbain sont également touchés (même s'il ne s'agit pas des mêmes violences). Pendant cette étude, nous avons recueilli pendant 4 ans les témoignages de 1500 jeunes en situation de grande difficulté, âgés de 8 à 15 ans. Nous avons recherché les causes de violence, à partir desquelles 9 portraits d'adolescents ayant des comportements violents les plus à risque ont été dégagés. Les voici, je vous invite à les transposer en fonction de votre pratique de parent ou d'enseignant.
  \end{itemize}

% --------------------------------------------------------------------
\subsection{1\textsuperscript{er} portrait: le manque de confiance en soi }
% --------------------------------------------------------------------

  100\,\% des enfants interrogés présentaient les mêmes caractéristiques: le doute, la culpabilité et le manque de confiance en soi. Il s'agit d'enfants ayant été dénigrés en permanence dans leur famille, ou mis en compétition défavorable avec un frère ou une sœur, ou trop responsabilisés (enfants « parentalisés »).\newline
  Insistons sur le danger qu'il y aurait à ne jamais faire comprendre à un enfant qu'il a le droit à l'échec. Dans cette zone fragile qu'est l'adolescence, s'ils n'ont pas été armés, s'ils ont l'impression que pour être aimés, ils doivent être les meilleurs, comparés à d'autres, ils seront particulièrement vulnérables au moindre échec.

% --------------------------------------------------------------------
\subsection{2\textsuperscript{ème} portrait: l'enfant roi }
% --------------------------------------------------------------------

  Un grand nombre des enfants rencontrés étaient des enfants rois ou chefs de famille. Ceux-là font tout, et le plus longtemps possible, pour rester dans le principe de plaisir.\newline
  Dès la petite enfance, l'enfant développe des stratégies pour rester le centre du monde (par exemple: tout faire pour dormir dans le lit de papa et maman, et à force de persévérance, réussir à garder maman pour lui tout seul). Résultat: bon nombre de séparations et divorces suivent la première année de la naissance d'un premier enfant. Qu'on se le dise: le plus grand service qu'un père puisse rendre à ses enfants, c'est de s'occuper de sa femme\dots\ Nous parlons là de la socialisation qui se met en marche.\newline
  Ces adolescents de 8 à 15 ans, mis sur un piédestal par leurs parents, n'ont pas eu les moyens d'accepter de ne pas être tout.

% --------------------------------------------------------------------
\subsection{3\textsuperscript{ème} portrait: la notion de temps }
% --------------------------------------------------------------------

  Beaucoup des adolescents concernés ne maîtrisaient pas la notion de temps.\newline
  En voici l'illustration:\newline
  \textit{« Pourquoi tu casses ? --- Parce qu'on s'ennuie.}\newline
  \textit{Pourquoi tu t'ennuies ? --- Parce qu'il n'y a rien pour nous.}\newline
  \textit{Qu'est-ce que tu voudrais ? --- On ne sait pas. »}\newline
  C'est à partir de ces données que nous nous sommes demandé pourquoi autant de jeunes ne savent pas s'occuper tout seul ? Deux profils se présentent.\newline
  Soit ils n'ont jamais eu de temps libre: on a retrouvé ce problème dès la petite enfance (exemples de crèches « super actives » ou/et de parents qui gèrent et remplissent intégralement le temps entre l'école, les activités extra scolaires, la maison\dots). Nous vivons dans un système trop basé sur la compétition qui nous fait croire que, si l'enfant ne fait rien, il perd son temps. Nous avons donc de vraies questions à nous poser par rapport à nos jeunes: \textit{« Est-ce que nous leur laissons du temps libre ? »}.\newline
  Soit ils n'ont jamais eu de temps captif, c'est à dire pas d'obligations (pas de rythmes imposés pour les heures du coucher, ou de scolarisation régulière\dots). Cela produit des enfants « zapping » qui ont besoin de changer d'activité à tout instant, et qui, dès l'école maternelle, vont perturber la classe pour attirer en permanence l'attention de la maîtresse. Nous retrouvons la question fondamentale du rapport au temps. Qu'il s'agisse de temps libre, de zapping ou d'incapacité à se projeter, tout commence dès la petite enfance: il faut pouvoir dire non à l'enfant pour lui apprendre à attendre. Dans un collège, en préparant quelque chose pour la fin de l'année, une sortie, une fête, on fait une action maintenant pour un plaisir plus tardif, on travaille énormément sur la notion de projet, et les jeunes ont besoin d'être impliqués dans un projet.

% --------------------------------------------------------------------
\subsection{4\textsuperscript{ème} portrait: les centres d'intérêt }
% --------------------------------------------------------------------

  Enfants n'ayant aucun pôle d'intérêt: on a constaté un décalage permanent entre les goûts de ces enfants et ce que les parents leur ont imposé. Ils ont été enfermés ou spécialisés dans certaines activités que les parents ont jugées « bonnes » pour eux, ils ont très vite appris qu'ils n'avaient pas le droit d'avoir envie de quelque chose.\newline
  On rencontre ce phénomène dans l'organisation des colonies de vacances\dots\ on ne dit plus « colonies de vacances », on dit « stages à thème »: la notion de « vacance » n'est plus vendable, il faut du thème, de la production, de la compétition\dots\newline
  Sachons que préparer l'adolescence, c'est développer des centres d'intérêt dès la petite enfance. Accompagner les adolescents, c'est aussi partager avec eux nos propres pôles d'intérêt, au-delà ou à travers l'enseignement.

% --------------------------------------------------------------------
\subsection{5\textsuperscript{ème} portrait: les rythmes de développement }
% --------------------------------------------------------------------

  Il s'agit d'enfants dont les rythmes biologiques ou de développement ont été inféodés aux rythmes des adultes et n'ont pas été respectés (journées trop chargées, enfants levés trop tôt, manque de sommeil\dots).\newline
  Parmi ces enfants aux rythmes perturbés, bon nombre ont souffert du « tout école »: centres de loisirs à l'école, temps de restauration à l'école\dots\ Quand je dis \textit{« L'école n'est pas responsable de tout », }je dois ajouter: \textit{« L'école ne doit pas répondre à tout »}.

  Au sujet des rythmes et du « tout école », mettons en parallèle l'expérience française et l'expérience québécoise. Dans ces deux pays, les femmes se sont mises à travailler hors foyer à peu près à la même période, et des problèmes de prise en charge des enfants le midi se sont alors posés.\newline
  En France, on a créé des cantines, et pour faire face au grand nombre d'enfants, on a mis en place plusieurs services sur un temps court (et les enfants avaient à peine le temps de manger\dots). Puis, comme il y avait trop d'enfants, on a accusé les parents d'être démissionnaires et on a décidé d'interdire la cantine aux enfants dont les parents ne travaillent pas\dots\ Grossière erreur, car beaucoup d'enfants dont les parents ne travaillent pas, auraient pour de multiples raisons, intérêt à rester à la cantine ! Inversement, d'autres enfants dont les parents travaillent, ne sont pas obligés de rester à la cantine.\newline
  Au Québec, on a essayé de trouver d'autres réponses, dans le registre du lien social. Quand les femmes se sont mises à travailler hors foyer, les élus ont financé des associations de personnes en retraite pour venir chercher deux à trois enfants à l'école le midi, et les emmener manger à la maison. On a donc créé des liens inter générationnels, on a lutté contre le sentiment d'insécurité car à travers cette action, on a fait en sorte que les uns n'aient pas peur des autres, et cela change tout: \textit{« On a transformé le problème en un plus »}. Alors qu'en France, on a tendance à transformer un problème en un autre problème. Seulement quelques élus courageux commencent à tenter ce genre d'expérience dans notre pays.

  Revenons aux rythmes de développement des enfants: que nous dit la recherche aujourd'hui ? Plus la recherche mondiale évolue, plus elle élargit la notion de normalité. Par exemple, les chercheurs nous montrent actuellement que la propreté s'acquiert entre 2 et 6 ans. Il est normal qu'un enfant soit propre à 2 ans, s'il y est prêt. Il est normal qu'il ne soit pas propre à 5 ans, s'il n'y est pas prêt. Ce qui ne présume en rien de la suite.\newline
  Si un enfant de 5 ans n'est pas propre, le médecin vérifiera s'il n'y a pas de problème organique. Puis, s'il est sérieux, il vous dira que vous ne pouvez rien faire pour le mettre en avance, mais par contre, que vous pouvez tout faire pour le mettre en retard\dots\ Voir l'exemple du Japon considérant que plus un enfant allait tôt à l'école, plus il aurait de chances de réussir plus tard. Dès l'âge de 18 mois, les enfants devaient être propres pour pouvoir entrer dans des écoles de sur-stimulation précoce. En version française: s'il est propre de bonne heure, il ira tôt à la maternelle, il saura donc lire avant le Cours Préparatoire, et il aura une bonne 6\textsuperscript{ème }d'allemand\dots\ On a donc vu énormément de violence chez des adolescents dont on n'avait pas pris en compte le rythme de développement.\newline
  La propreté: 2 à 6 ans. La lecture ? Les travaux les plus avancés indiquent l'apprentissage de la lecture possible entre 4 et 9 ans. Il est normal qu'un enfant de 4 ans sache lire, s'il y est prêt. Il est tout aussi normal qu'il ne maîtrise pas la lecture à 8 ans, s'il n'y est pas prêt. Ce qui en soit ne présume en rien de ses futurs talents de lecteur, ni de ses compétences dans d'autres diplômes\dots\ Sauf à l'intérieur d'un système éducatif où il est marginalisé: comment cet enfant pourra-t-il se construire s'il est considéré comme nul ?\dots\ (voir chapitre précédent sur la confiance).

% --------------------------------------------------------------------
\subsection{6\textsuperscript{ème} portrait: la notion de territoire }
% --------------------------------------------------------------------

  Ces adolescents violents n'ont pas acquis la notion de territoire, ils sont restés dans le principe de plaisir de la petite enfance. Cela se jouait de la même façon au jardin public sur le tas de sable: \textit{« J'ai tout ce que je veux, tout est à moi »}. Sauf que les petits enfants y apprennent la socialisation: \textit{« Je pique le seau de celui-là\dots\ Il ne se laisse pas faire et me tape\dots\ Je pleure, le temps qu'il me sera nécessaire pour construire cette certitude: le monde entier est à moi\dots\ sauf mon seau »}. A condition que les parents ne soient pas intervenus trop rapidement\dots\ Parenthèse: on a vu beaucoup de phénomènes d'auto-violence chez les adolescents qui n'avaient jamais appris à gérer les conflits; et qui avaient toujours eu des adultes pour les gérer à leur place.\newline
  Remplaçons donc le seau par une montre, l'âge des enfants qui n'ont plus 3 ans mais 13 ans, et changeons le lieu en passant du square à la cours du collège, et cela devient du racket. Un tiers des jeunes auteurs de racket rencontrés dans notre étude en étaient encore au stade du seau: ils n'ont pas bien compris ce qu'ils faisaient, et ne pouvaient pas imaginer une seconde que tout n'était pas à eux. La question à se poser est la suivante: où en sont-ils par rapport à la loi ?\newline
  Toutes les données se croisent car, au-delà de la notion de territoire, ils n'ont pas intégré la loi. Quelquefois, il y a chez les adolescents, des choses à apprendre de la petite enfance (voir plus haut « sessions de rattrapage »), Mais il faut aussi apprendre aux parents à cadrer les enfants ; ces parents qui sont les premiers éducateurs, auxquels il faut donner confiance pour qu'ils puissent dire non dès la petite enfance.

% --------------------------------------------------------------------
\subsection{7\textsuperscript{ème} portrait: la connaissance de l'histoire familiale }
% --------------------------------------------------------------------

  Ces jeunes ont manqué de repères par rapport à leur histoire familiale, à leur identité. Une multitude d'exemples peut illustrer ce phénomène.\newline
  J'ai participé à une étude sur les dysfonctionnements scolaires constatés en Seine Saint-Denis chez des enfants de harkis. L'origine de leurs difficultés a été trouvée dans le manque de lisibilité de leurs deux origines: d'un côté, les parents leur interdisaient de jouer avec les arabes à l'école, et d'un autre, leurs camarades ne les considéraient pas comme français. Ces enfants se dessinaient en deux morceaux. La problématique qui apparaissait n'était pas la double culture, mais l'interdiction de séjour dans les deux cultures. Comment un enfant qui perçoit son corps morcelé peut-il apprendre la géométrie, tracer une ligne droite, écrire correctement ? Un travail a dû être engagé scolairement pour recoller les morceaux en valorisant les deux cultures.\newline
  Autre situation qui interroge l'enfant sur son histoire et son identité: elle concerne les enfants nés par procréation médicalement assistée et qui ont deux pères. Ils ont un travail de deuil à faire en terme de lisibilité pour pouvoir construire une relation avec un homme qui n'est pas leur père biologique, mais qui est leur père aimant. La question se pose en Suède de façon cruciale car une loi de 1982 a rendu le don de sperme non anonyme, et a donné le droit aux enfants de connaître le nom du donneur dès l'âge de 8 ans. On a par la suite observé une augmentation du nombre de tentatives de suicides chez les jeunes, notamment ceux qui apprenaient d'un seul coup une histoire ingérable.\newline
  Voici d'autres exemples puisés dans des familles monoparentales et correspondant aux manques de repères familiaux. Au dernier recensement, à peine plus de deux millions d'enfants sont élevés par un seul parent, dont plus de 80\,\% par la mère. Est-ce constructible ? Oui, à condition que l'histoire de cet enfant soit lisible.\newline
  Si une maman dit: \textit{« Il n'a pas de père »}, c'est illisible pour lui. Tout enfant a un père, et cet enfant ne pourra pas se construire en ayant l'impression d'un vide. Il faudra donc aider cette maman à pouvoir mettre des mots (ou se faire aider pour en mettre) sur ce père, qui sera un papa « matériau » pour la construction de l'enfant. Il ne sera pas « matériel » car il ne sera pas là, mais il deviendra « matériau » au niveau de l'imaginaire: un prénom, une photographie,\dots\ quelque soit l'élément fourni pour permettre au jeune de se construire.\newline
  Citons l'exemple d'une autre mère élevant seule son enfant, et dénigrant sans cesse le père. C'est très lisible pour un enfant d'entendre disqualifier un parent. Par contre, c'est inconstructible. Beaucoup d'enfants violents dans les cours de collèges souffrent d'être identifiés à un parent négatif. Il faudra dans ce cas, amener cette maman à ne pas confondre sa querelle de femme avec l'image qu'elle véhicule de ce papa. Il est essentiel que dans les couples séparés, l'homme et la femme sachent faire exister le père et la mère d'une façon positive dans la tête de l'enfant.\newline
  Dans les années cinquante, le professeur Léon Kreysler a étudié le développement pondéral dans les familles dont les parents disaient: \textit{« On se séparera quand les enfants seront grands »}. Ses travaux ont donné naissance à un syndrome portant son nom qui s'appelle le « nanisme social ». Il a montré que certains enfants pouvaient aller jusqu'à arrêter leur processus de croissance pour maintenir la famille en état. Voici encore une démonstration des conséquences possibles du problème identitaire (à ce sujet, voir le film de Volker Schlöndorf intitulé \textit{« Le Tambour »}),

% --------------------------------------------------------------------
\subsection{8\textsuperscript{ème} portrait: la conscience de la portée des actes }
% --------------------------------------------------------------------

  Il s'agit d'enfants auteurs de violence n'ayant pas compris la portée de leurs actes: ils ne font pas la différence entre le virtuel et le réel, ils ne comprennent pas le sens de ce qu'ils font. On a constaté énormément de violence chez ces jeunes restés au stade du tout petit, du monde virtuel, victimes de pathologies de plus en plus fréquentes.\newline
  Elles peuvent être causées par un usage permanent et sans accompagnement de la télévision, des consoles vidéo, ou des jeux de rôles. Une étude relativement récente a montré que les scènes violentes à la télévision sont plus dures à vivre pour des jeunes quand elles sont regardées en famille si les parents n'en parlent pas, car de cette façon, ils cautionnent ce qui est vu. Toutefois, je ne fais pas le procès des consoles vidéo qui peuvent sans doute servir d'outils sensationnels dans les rapports adultes/adolescents.\newline
  Cependant, si ces supports sont utilisés pour échapper à une réalité, ils peuvent provoquer des comportements de violence graves assimilés par les jeunes à des jeux sans conséquence.

% --------------------------------------------------------------------
\subsection{9\textsuperscript{ème} portrait: la notion de mort et de vivant }
% --------------------------------------------------------------------

  En lien direct avec le portrait précédent, ce dernier concerne des adolescents violents qui ont été mis complètement à l'écart de la mort. L'apprentissage de la mort et du vivant font en effet partie des apprentissages fondamentaux. Cette pathologie n'existait pas dans les années soixante car la France était à 80\,\% rurale ou semi-rurale, et les enfants assistaient à la mise à mort d'animaux d'élevage. Dans notre société, la question ne s'était encore jamais posée parce que, de tout temps, les gens mouraient plus dans le cadre familial, et les jeunes étaient mêlés à des rituels.\newline
  Aujourd'hui, ces jeunes qui n'ont ni compris la portée de leurs actes, ni intégré la notion de mort et de vivant, sont de véritables bombes. Ce sont précisément ceux-là que nous avons rencontrés en milieu carcéral.


% =============================================================
\section{CONCLURE }
% =============================================================

Chaque portrait décrit ici correspond à la non acquisition de 9 apprentissages fondamentaux au sens de fondation pour la construction de l'enfant. Il ne s'agit pas des apprentissages fondamentaux de l'école primaire: la lecture, l'écriture, le calcul. C'est réellement « une trousse de survie » pour la suite. Alors, comment conclure ?\newline
Que l'ont soit parent ou adulte extérieur à la famille, malgré les situations difficiles vécues ou observées, il faut garder présent à l'esprit que rien n'est jamais perdu, des « sessions de rattrapage » sont toujours possibles. Il existe toujours des adultes, des professeurs, des associatifs\dots\ qui peuvent regarder le jeune différemment, et provoquer chez lui le déclic lui permettant de se réinvestir, retrouver confiance en lui, et réussir dans son parcours.

\end{document}
